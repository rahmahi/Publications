\documentclass[aps,prb,reprint,showpacs,floatfix,superscriptaddress, onecolumn, nofootinbib, 10pt]{revtex4-2}

\usepackage{amsmath,amsthm,amssymb}
\usepackage{graphicx}% Include figure files
\usepackage{dcolumn}% Align table columns on decimal point
\usepackage{bm}% bold math
\usepackage{color}
\usepackage{epsfig}
\usepackage{multirow}
\usepackage{mathrsfs}
\usepackage{hyperref}
\usepackage{cleveref}
\usepackage{epstopdf}
\usepackage{subfigure}
\usepackage{autobreak}
\usepackage{todonotes}
\usepackage{physics}
\usepackage{bbm}
\usepackage[normalem]{ulem}


\usepackage[absolute,overlay]{textpos}

%Macros for mathematical notations

\newcommand{\V}[1]{\boldsymbol{#1}} %# vector
\newcommand{\M}[1]{\boldsymbol{#1}} %# matrix
\newcommand{\Set}[1]{\mathbb{#1}} %# set
\newcommand{\D}[1]{\Delta#1} %# \D{t} for time step size
\renewcommand{\d}[1]{\delta#1} %# \d{t} for small increment
\newcommand{\av}[1]{\left\langle #1\right\rangle } %take average

\newcommand{\sM}[1]{\M{\mathcal{#1}}} %matrix in mathcal font
\newcommand{\dprime}{\prime\prime} % double prime
%\global\long\def\i{\iota}
%\renewcommand{\i}{\iota} %i for imaginary unit
%\renewcommand{\i}{\mathsf i} %i for imaginary unit
\newcommand{\follows}{\quad\Rightarrow\quad} %=>
\newcommand{\eqd}{\overset{d}{=}} %=^d
\newcommand{\spe}[1]{\mathscr{#1}}  %important quantities in mathscr font
\newcommand{\eps}{\epsilon}

\newcommand{\response}[1]{{\color{black}#1}} % for authors' response
\newcommand{\comment}[1]{{\color{blue}#1}} % for referee's comment


\title{reply to Mori}

\begin{document}
The numerical results, the equations in our manuscript, and our author response have all been rechecked,

Through out the manuscript, it should be noted that instead of summing over ij index pairs, we have been summing over $i-j$ bonds.

Once we've collected the trace-full contributions $i=k$, $j=l$, the choice $i-j$, $j=k$ becomes irrelevant insofar as the square of the $H_0$ is concerned. In order to account for over count, we would have to multiply $H_0$ by $1/2$ if we had started counting in pairs, and it'd cancel out with the "2" from the extra choice.

As a result, at the thermodynamic limit, the ultimate conclusions remains $J^2/8$.

We have therefore resolved this ambiguity in our response, along with a minor arithmetic error in the penultimate step. By carefully calculating $H_0^2$ on paper using Pauli matrices for N=4, we further confirmed our result (the six perfect squares are the only trace-full terms).

We consider $N=4$ and the corresponding Hamiltonian,

\begin{align}
H_0 &= \frac{J}{2(N-1)} \sum_{i \ne j} \hat{\sigma^z_i}\hat{\sigma^z_j}\nonumber\\
&= \frac{J}{2(N-1)} \Big[\hat{\sigma}^z_0 \hat{\sigma}^z_1 + \hat{\sigma}^z_0 \hat{\sigma}^z_2 + \hat{\sigma}^z_0 \hat{\sigma}^z_3 + \hat{\sigma}^z_1 \hat{\sigma}^z_2 + \hat{\sigma}^z_2 \hat{\sigma}^z_3 + \hat{\sigma}^z_3 \hat{\sigma}^z_1\Big]
\end{align}
Therefore,
\begin{align}
	H_0^2 &= \frac{J^2}{2^2(N-1)^2} \left[\sum_{i \ne j} \hat{\sigma^z_i}\hat{\sigma^z_j}\right]^2\nonumber\\
	&= \frac{J^2}{2^2(4-1)^2} \Big[\left(\hat{\sigma}^z_0 \hat{\sigma}^z_1\right)^2 + \left(\hat{\sigma}^z_0 \hat{\sigma}^z_2\right)^2  + \left(\hat{\sigma}^z_0 \hat{\sigma}^z_3\right)^2  + \left(\hat{\sigma}^z_1 \hat{\sigma}^z_2\right)^2  + \left(\hat{\sigma}^z_2 \hat{\sigma}^z_3\right)^2  + \left(\hat{\sigma}^z_3 \hat{\sigma}^z_1\right)^2  + ${ all other traceless terms}$\Big]\nonumber\\
	&= \frac{J^2}{2^2(4-1)^2} \big(\mathbbm{1} \times 6\big)
\end{align}
Thus, 
\begin{equation}
\frac{\Tr[H_0^2]}{\Tr[\mathbbm{1}]} = \frac{J^2}{4 \times 9} \times 6 = \frac{J^2}{6} .
\end{equation}
In case of general system size N,

\begin{align}
H_0 &= \frac{J}{2(N-1)} \sum_{ij} \hat{\sigma}^z_i \hat{\sigma}^z_j. \\
${Thus, }$\Tr[H_0^2] &= \frac{J^2}{4(N-1)^2}\Tr\left[\sum_{ij} \hat{\sigma}^z_i \hat{\sigma}^z_j\right]^2\nonumber\\
&= \frac{J^2}{4(N-1)^2}\; \mathbbm{1} \times\; \binom{N}{2}\nonumber\\
&=\frac{J^2}{4(N-1)^2}\frac{N(N-1)}{2}\nonumber\\
&=\frac{J^2}{8}\frac{N}{N-1}
\end{align}

In the thermodynamic limit, considering $N\to\infty$, $\displaystyle \Tr[H_0^2] \simeq \frac{J^2}{8}$.\\


I've enclosed revised drafts of the manuscript and response. Please review them and provide any feedback if any.
\end{document}