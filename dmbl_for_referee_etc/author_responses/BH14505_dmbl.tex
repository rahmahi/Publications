\documentclass[aps,prb,reprint,showpacs,floatfix,superscriptaddress, onecolumn, nofootinbib, 9pt]{revtex4-2}

\usepackage{amsmath,amsthm,amssymb}
\usepackage{graphicx}% Include figure files
\usepackage{dcolumn}% Align table columns on decimal point
\usepackage{bm}% bold math
\usepackage{color}
\usepackage{epsfig}
\usepackage{multirow}
\usepackage{mathrsfs}
\usepackage{hyperref}
\usepackage{cleveref}
\usepackage{epstopdf}
\usepackage{subfigure}
\usepackage{autobreak}

\usepackage{physics}
\usepackage{bbm}

%Macros for mathematical notations

\newcommand{\V}[1]{\boldsymbol{#1}} %# vector
\newcommand{\M}[1]{\boldsymbol{#1}} %# matrix
\newcommand{\Set}[1]{\mathbb{#1}} %# set
\newcommand{\D}[1]{\Delta#1} %# \D{t} for time step size
\renewcommand{\d}[1]{\delta#1} %# \d{t} for small increment
\newcommand{\av}[1]{\left\langle #1\right\rangle } %take average

\newcommand{\sM}[1]{\M{\mathcal{#1}}} %matrix in mathcal font
\newcommand{\dprime}{\prime\prime} % double prime
%\global\long\def\i{\iota}
%\renewcommand{\i}{\iota} %i for imaginary unit
%\renewcommand{\i}{\mathsf i} %i for imaginary unit
\newcommand{\follows}{\quad\Rightarrow\quad} %=>
\newcommand{\eqd}{\overset{d}{=}} %=^d
\newcommand{\spe}[1]{\mathscr{#1}}  %important quantities in mathscr font
\newcommand{\eps}{\epsilon}

\newcommand{\ar}[1]{{\color{blue}#1}} % for authors' response


\begin{document}
\preprint{Preprint}

\title{First Response to Referee Comments for Manuscript BH14505}
\author{Analabha Roy}
\date{\today}

\maketitle

\vspace{1em}

\noindent \textbf{Response to First Referee}

\begin{enumerate}
\item The referee says, ``\textit{A brief comparison of this work to other works on periodically driven LMG models and DMBL is needed somewhere in the introductory sections to state what is the significance and novelty of this work. }"\\

\ar{
We thank the referee for raising this issue. In earlier works, the observable(s) corresponding to the system Hamiltonian have been a keen matter of interest from which the dynamical many-body localization (DMBL) is investigated. Observables such as magnetization, anharmonicity, normalized excitation energy, participation ratio, quantum fidelity, and others have been considered. The observables are computed for an extended duration. It is found in both closed and open quantum systems.

The novelty in our paper resides in the introduction of the Floquet theory, which deals with the system at stroboscopic times during each time period. Floquet theory yields Floquet modes (FM), which bear out the exact dynamics of the system and are true at all times. This is the origin of the Floquet Eigen State Thermalization Theory (FETH). FM can mix themselves and exhibit thermal behavior at specific conditions. If the system parameters are properly controlled, one can prevent the mixing of the FM. FM can be applied to investigate the Inverse Participation Ratio, which distinguishes between a fully localized state and a distributed state of the system. IPR ranges from zero (a fully distributed or thermal state) to unity (a fully localized state). We have shown that, apart from other widely used observables, to calculate IPR, applying FETH and utilizing FM is a better window as it is valid for infinite times.
}
\item The referee says, ``\textit{On page 3, the Floquet eigenstate Thermalization hypothesis is stated without references. }".\\

\ar{
We thank the referee for pointing out this mistake. We have introduced proper reference against FETH in the manuscript.
}

\item The referee says, ``\textit{On page 4, after illustrating the resonances in the analytically solvable TFIM model, the authors claim, “This phenomenon is highly general and can be readily adapted to non-integrable systems”. I disagree with the statement, and clear references, if any, need to be cited to back up this statement. The manipulations made for TFIM are fine-tuned to this integrable system, and they break down the moment any integrability-breaking term is introduced. For example, if I add a longitudinal field sigma$_$x to the TFIM, I do not see how to adapt the procedure. }"\\

\ar{

}
\item The referee says,``\textit{In the discussion of phase crossover from thermal to DMBL, increasing N in Fig 8 appears to push the regime of the local phase to a larger drive frequency. This seems to raise the same concerns associated with the stability of disorder-induced localization, as in whether one needs infinite disorder or infinite driving frequency to
get a localization in the thermodynamic limit. Is this the case? }".

\ar{    	
We agree with the referee on this point. \\
{\bf For further argument MR needs guidance from AR.}		
}

\item The referee says,``\textit{Is the heating suppressed at points where the frequency meets the resonance condition? }".

\ar{
We thank the referee for this understanding. We agree with the referee. The resonance condition derived from the Rotating Wave Approximation(RWA) on the Hamiltonian of the system. At resonance condition(s), it is possible to nullify the effective Hamiltonian. This results in total localization of the system. This is true for infinite times. Thus the system at resonance condition never undergo the heating.
}

\item The referee says, ``\textit{The discussion related to Fig. 9 could be improved, and it is not clear to me why the standard deviation of temporal fluctuations should vanish for thermalizing systems. }"\\

\ar{

}

\item The referee says,``\textit{In the conclusion, it is useful to comment on whether this method could be adapted (related to my point 2) to other generic models (say
	XXZ models). }"\\

\ar{

}
\vskip 1cm 
\noindent \textbf{Summary of important changes to the  manuscript}


\begin{enumerate}
\item 
\end{enumerate}

\bibliography{dmbl_refs}


\end{document}
