\documentclass[a4paper,11pt]{article}
\usepackage[left=1cm, right=1cm, top=1cm, bottom=1.5cm]{geometry} 
\usepackage{amsmath}
\usepackage{amsthm}
\usepackage{amssymb}
\usepackage[sort&compress]{natbib}
\usepackage{ulem}
\usepackage{url}
\usepackage{hyperref}
\usepackage{bbm}
\usepackage{dcolumn}

\usepackage[dvipsnames]{xcolor}
\newcommand{\red}[1]{\textcolor{red}{#1}}
\newcommand{\blue}[1]{\textcolor{blue}{#1}}
\usepackage{cancel}

\title{Ising Model}

\author{}
\date{}


\begin{document}
\maketitle
\section{Transverse Field Ising Model (TFIM)}
The nearest neighbour Ising model corresponds to Hamiltonian,

\begin{align}
\hat{\mathcal{H}}(t) & =\frac{1}{2}\left[\sum_{i} J \hat{\sigma}_{i}^{x} \hat{\sigma}_{i+1}^{x}+h \cos (\omega t) \sum_{i} \hat{\sigma}_{i}^{z}\right] \\
H_{0} & =\frac{1}{2} \sum_{i} J \hat{\sigma}_{i}^{x} \hat{\sigma}_{i+1}^{x} \\
H_{1} & =\frac{1}{2} h \cos (\omega t) \sum_{i} \hat{\sigma}_{i}^{z}
\end{align}

The unitary evolution operator $\hat{U}(t)$ is,


\begin{align}
\hat{U}(t) & =\exp \left(-i \int_{0}^{t} \frac{h}{2} \cos (\omega t) \sum_{i} \hat{\sigma}_{i}^{z} d t\right) \nonumber\\
& =\exp \left(-i \frac{h}{2 \omega} \sin (\omega t) \sum_{i} \hat{\sigma}_{i}^{z}\right) \\
& =\prod_{i} \exp \left(-i \frac{h}{2 \omega} \sin (\omega t) \hat{\sigma}_{i}^{z}\right) .
\end{align}


The generic moving frame effective Hamiltonian for $\hat{\mathcal{H}}(t)$,

\begin{equation}
\tilde{H}(t)=\hat{U}^{\dagger}(t) \hat{\mathcal{H}}(t) \hat{U}(t)-i \hat{U}^{\dagger}(t) \partial_{t} \hat{U}(t)
\label{eq:ht}
\end{equation}

Lets find out first the partial derivative of unitary transform operator,

\begin{equation}
-i \partial_{t} \hat{U}(t)=\prod_{i}\left(-\frac{h}{2} \cos (\omega t) \hat{\sigma}_{i}^{z}\right) \exp \left(-i \frac{h}{2 \omega} \sin (\omega t) \hat{\sigma}_{i}^{z}\right)
\end{equation}

Then,

\begin{align}
-i \hat{U}^{\dagger}(t) \partial_{t} \hat{U}(t)= & \prod_{k}\left\{\exp \left(i \frac{h}{2 \omega} \sin (\omega t) \hat{\sigma}_{k}^{z}\right)\right\} \prod_{i}\left\{\left(-\frac{h}{2} \cos (\omega t) \hat{\sigma}_{i}^{z}\right) \exp \left(-i \frac{h}{2 \omega} \sin (\omega t) \hat{\sigma}_{i}^{z}\right)\right\} ; i=k
\label{eq:eqpartial}
\end{align}

Using Eq.\eqref{eq:eqpartial}, Eq.\eqref{eq:ht} becomes,


\begin{align*}
\hat{\mathcal{H}}(t) & =\hat{U}^{\dagger}(t) \frac{1}{2}\left(\sum_{i} J \hat{\sigma}_{i}^{x} \hat{\sigma}_{i+1}^{x}+h \cos (\omega t) \sum_{i} \hat{\sigma}_{i}^{z}\right) \hat{U}(t)-\sum_{i}\left(\frac{h}{2} \cos (\omega t) \hat{\sigma}_{i}^{z}\right) \nonumber\\
& =\frac{1}{2} \hat{U}^{\dagger}(t)\left(\sum_{i} J \hat{\sigma}_{i}^{x} \hat{\sigma}_{i+1}^{x}\right) \hat{U}(t)+\sum_{i}\left(\frac{h}{2} \cos (\omega t) \hat{\sigma}_{i}^{z}\right)-\sum_{i}\left(\frac{h}{2} \cos (\omega t) \hat{\sigma}_{i}^{z}\right) \nonumber\\
& =\frac{1}{2} \hat{U}^{\dagger}(t)\left(\sum_{i} J \hat{\sigma}_{i}^{x} \hat{\sigma}_{i+1}^{x}\right) \hat{U}(t) \nonumber\\
& =\frac{1}{2} \exp \left(i \frac{h}{2 \omega} \sin (\omega t) \sum_{i} \hat{\sigma}_{i}^{z}\right)\left(\sum_{i} J \hat{\sigma}_{i}^{x} \hat{\sigma}_{i+1}^{x}\right) \exp \left(-i \frac{h}{2 \omega} \sin (\omega t) \hat{\sigma}_{i}^{z}\right)\nonumber
\end{align*}

We utilize $\eta=\frac{h}{2 \omega} \sin (\omega t)$, also considering $J=1$.


\begin{align}
\hat{\mathcal{H}}(t) & =\frac{1}{2} \exp \left(i \frac{h}{2 \omega} \sin (\omega t) \sum_{i} \hat{\sigma}_{i}^{z}\right)\left(\sum_{i} J \hat{\sigma}_{i}^{x} \hat{\sigma}_{i+1}^{x}\right) \exp \left(-i \frac{h}{2 \omega} \sin (\omega t) \hat{\sigma}_{i}^{z}\right) \nonumber\\
& \simeq \prod_i \exp\left(i\eta\hat{\sigma}^z_i\right)\left(\sum_{i} J \hat{\sigma}_{i}^{x} \hat{\sigma}_{i+1}^{x}\right)\exp\left(-i\eta\hat{\sigma}^z_i\right)\nonumber\\
&= \sum_i J \exp(i\eta\hat{\sigma}^z_i)\exp(i\eta\hat{\sigma}^z_{i+1})\left( \hat{\sigma}_{i}^{x} \hat{\sigma}_{i+1}^{x}\right)\exp(-i\eta\hat{\sigma}^z_i)\exp(-i\eta\hat{\sigma}^z_{i+1})\nonumber\\
&= J\sum_{i} \hat{\sigma}_{i}^{x} \hat{\sigma}_{i+1}^{x}\exp(-2i\eta\hat{\sigma}^z_i)\exp(-2i\eta\hat{\sigma}^z_{i+1})
\label{eq:hmov}
\end{align}

Using, $\displaystyle e^{ia(\hat{n}\cdot \vec{\sigma})} = \mathbbm{1} \cos(a) + i (\hat{n}\cdot \vec{\sigma})\sin(a)$, 
\begin{align}
\exp\left(-2i\eta\hat{\sigma}^z_i\right)\exp\left(-2i\eta\hat{\sigma}^z_i\right)
=& \Big[\mathbbm{1} \cos(2\eta) - i \hat{\sigma}^z_i\sin(2\eta)\Big]\Big[\mathbbm{1} \cos(2\eta) - i \hat{\sigma}^z_{i+1}\sin(2\eta)\Big]\nonumber\\
=& \cos^2(2\eta) - \hat{\sigma}^z_i\hat{\sigma}^z_{i+1}\sin^2(2\eta) -\frac{i}{2} \left(\hat{\sigma}^z_i+ \hat{\sigma}^z_{i+1}\right)\sin(4\eta)
\label{eq:op}
\end{align}

\begin{align}
\cos^2(2\eta) =& \frac12\big[1+\cos(4\eta)\big]=\frac12\big[\cos^2(4\eta)+ \sin^2(4\eta)+\cos(4\eta)\big]\nonumber\\
\sin^2(2\eta) =& \frac12\big[1-\cos(4\eta)\big]=\frac12\big[\cos^2(4\eta)+ \sin^2(4\eta)-\cos(4\eta)\big]
\label{eq:cossin}
\end{align}

Now, we recall Jacobi Anger expansion,


\begin{align}
\cos (4 \eta)&=\cos \left(\frac{2 h}{\omega} \sin (\omega t)\right)=\mathcal{J}_{0}\left(\frac{2 h}{\omega}\right)+2 \sum_{n=1}^{\infty}(-1)^{n} \mathcal{J}_{2 n}\left(\frac{2 h}{\omega}\right) \cos (2 n \omega t), \\
\sin (4 \eta)&=\sin \left(\frac{2 h}{\omega} \sin (\omega t)\right)=-2 \sum_{n=1}^{\infty}(-1)^{n} \mathcal{J}_{2 n-1}\left(\frac{2 h}{\omega}\right) \cos [(2 n-1) \omega t]
\label{eq:jacang}
\end{align}


here $\mathcal{J}_{n}$ is the Bessel function of n'th order. Using Rotating Wave Approximation(RWA) we can neglect the highrer order terms in Jacobi Anger expansion in Eq.\eqref{eq:jacang}. Thus considering $n=0, \cos (4 \eta) \simeq$ $\mathcal{J}_{0}\left(\frac{2 h}{\omega}\right)$, and $\sin (4 \eta) \simeq 0$, Eq.\eqref{eq:op}

\begin{align}
\cos^2(2\eta) \simeq& \frac12\Bigg[\left\{\mathcal{J}_0\left(\frac{2h}{\omega}\right)\right\}^2 + \mathcal{J}_0\left(\frac{2h}{\omega}\right)\Bigg]=\frac12 \mathcal{J}_0\left(\frac{2h}{\omega}\right)\Bigg[1+ \mathcal{J}_0\left(\frac{2h}{\omega}\right)\Bigg]\nonumber\\
\sin^2(2\eta) =& \frac12\Bigg[\left\{\mathcal{J}_0\left(\frac{2h}{\omega}\right)\right\}^2 - \mathcal{J}_0\left(\frac{2h}{\omega}\right)\Bigg]=\frac12 \mathcal{J}_0\left(\frac{2h}{\omega}\right)\Bigg[1- \mathcal{J}_0\left(\frac{2h}{\omega}\right)\Bigg]
\label{eq:cossin}
\end{align}
Thus,
\begin{equation}
\exp\left(-2i\eta\hat{\sigma}^z_i\right)\exp\left(-2i\eta\hat{\sigma}^z_i\right) = \frac12 \mathcal{J}_0\left(\frac{2h}{\omega}\right)\Bigg[\left\{1+ \mathcal{J}_0\left(\frac{2h}{\omega}\right)\right\} - \hat{\sigma}^z_i\hat{\sigma}^z_{i+1}\left\{1- \mathcal{J}_0\left(\frac{2h}{\omega}\right)\right\} \Bigg]
\end{equation}
\begin{equation}
\hat{\mathcal{H}}_{_{TFIM}}^{R W A}=\frac{J}{2} \mathcal{J}_0\left(\frac{2h}{\omega}\right)\Bigg[\left\{1+ \mathcal{J}_0\left(\frac{2h}{\omega}\right)\right\} - \hat{\sigma}^z_i\hat{\sigma}^z_{i+1}\left\{1- \mathcal{J}_0\left(\frac{2h}{\omega}\right)\right\} \Bigg]\sum_{i} \hat{\sigma}_{i}^{x} \hat{\sigma}_{i+1}^{x} 
\end{equation}

When drive parameter $\frac{2 h}{\omega}$ is one of the roots of Bessels function, the moving frame Hamiltonian (Eq. (10) becomes zero. This manifests in localization in the TFIM model.

\section*{2 Transverse Field Ising Model: TFIM $+\sigma_{x}$}
The nearest neighbour Ising model corresponds to Hamiltonian,

\begin{align}
\hat{\mathcal{H}}(t) & =\frac{1}{2}\left[\sum_{i} J \hat{\sigma}_{i}^{x} \hat{\sigma}_{i+1}^{x}+\sum_{i} \hat{\sigma}_{i}^{x}+h \cos (\omega t) \sum_{i} \hat{\sigma}_{i}^{z}\right]\\
H_{0} & =\frac{1}{2} \sum_{i} J \hat{\sigma}_{i}^{x} \hat{\sigma}_{i+1}^{x}+\sum_{i} \hat{\sigma}_{i}^{x} \nonumber\\
H_{1} & =\frac{1}{2} h \cos (\omega t) \sum_{i} \hat{\sigma}_{i}^{z}\nonumber
\end{align}

The unitary evolution operator $\hat{U}(t)$ is, $\hat{U}(t)=\prod_{i} \exp \left(-i \frac{h}{2 \omega} \sin (\omega t) \hat{\sigma}_{i}^{z}\right)$. Now the Hamiltonian in rotating frame,

\begin{align}
\tilde{\mathcal{H}}(t)= & \frac{1}{2} \exp \left(i \frac{h}{2 \omega} \sin (\omega t) \sum_{i} \hat{\sigma}_{i}^{z}\right)\left(\sum_{i} J \hat{\sigma}_{i}^{x} \hat{\sigma}_{i+1}^{x}+\sum_{i} \hat{\sigma}_{i}^{x}\right) \exp \left(-i \frac{h}{2 \omega} \sin (\omega t) \hat{\sigma}_{i}^{z}\right) \nonumber\\
= & \underbrace{\frac{1}{2} \exp \left(i \frac{h}{2 \omega} \sin (\omega t) \sum_{i} \hat{\sigma}_{i}^{z}\right)\left(\sum_{i} J \hat{\sigma}_{i}^{x} \hat{\sigma}_{i+1}^{x}\right) \exp \left(-i \frac{h}{2 \omega} \sin (\omega t) \hat{\sigma}_{i}^{z}\right)}_{\mathrm{A}} \nonumber\\
& +\underbrace{\frac{1}{2} \exp \left(i \frac{h}{2 \omega} \sin (\omega t) \sum_{i} \hat{\sigma}_{i}^{z}\right)\left(\sum_{i} \hat{\sigma}_{i}^{x}\right) \exp \left(-i \frac{h}{2 \omega} \sin (\omega t) \hat{\sigma}_{i}^{z}\right)}_{\mathrm{B}}
\end{align}

Let us find $A$ and $B$ separately. We already have solved $A$ in previous section in Eq. $(10)$, so we need to solve only $B$ part,


\begin{align*}
B & =\frac{1}{2} \exp \left(i \frac{h}{2 \omega} \sin (\omega t) \sum_{i} \hat{\sigma}_{i}^{z}\right)\left(\sum_{i} \hat{\sigma}_{i}^{x}\right) \exp \left(-i \frac{h}{2 \omega} \sin (\omega t) \hat{\sigma}_{i}^{z}\right) \\
& =\left(e^{i 2 \eta S^{z}} S^{x} e^{-i 2 \eta S^{z}}\right)
\end{align*}

Again applying Jacobi Anger expansion

\begin{align*}
& \cos (2 \eta)=\cos \left(\frac{h}{\omega} \sin (\omega t)\right)=\mathcal{J}_{0}\left(\frac{h}{\omega}\right)+2 \sum_{n=1}^{\infty}(-1)^{n} \mathcal{J}_{2 n}\left(\frac{h}{\omega}\right) \cos (2 n \omega t), \\
& \sin (2 \eta)=\sin \left(\frac{h}{\omega} \sin (\omega t)\right)=-2 \sum_{n=1}^{\infty}(-1)^{n} \mathcal{J}_{2 n-1}\left(\frac{h}{\omega}\right) \cos [(2 n-1) \omega t],
\end{align*}

and considering RWA which keeps only $n=0$ 'th term,

\begin{align}
B & =\left(e^{i 2 \eta S^{z}} S^{x} e^{-i 2 \eta S^{z}}\right) \nonumber\\
& \simeq\left(S^{x} \cos (2 \eta)-S^{y} \sin (2 \eta)\right) \simeq \mathcal{J}_{0}\left(\frac{h}{\omega}\right) S^{x}
\label{eq:B}
\end{align}

Thus we get,

\begin{equation}
\hat{\mathcal{H}}_{_{TFIM+S_{x}}}^{R W A}=\frac{J}{2} \mathcal{J}_0\left(\frac{2h}{\omega}\right)\Bigg[\left\{1+ \mathcal{J}_0\left(\frac{2h}{\omega}\right)\right\} - \hat{\sigma}^z_i\hat{\sigma}^z_{i+1}\left\{1- \mathcal{J}_0\left(\frac{2h}{\omega}\right)\right\} \Bigg]\sum_{i} \hat{\sigma}_{i}^{x} \hat{\sigma}_{i+1}^{x} +\mathcal{J}_{0}\left(\frac{h}{\omega}\right) S^{x}
\label{eq:tfimpsx}
\end{equation}

Now in the above Eq.\eqref{eq:tfimpsx} to find out the drive parameter which can make it zero, is difficult to find, because it consists of two different Bessel function. So, introducing an additional integrability breaking term `` $\sigma_{x}$ " in TFIM equation is not general practice in process to obtain localization.

\section{XXZ model}
Generic Heisenberg model
\begin{equation*}
H = \frac12 \left( \sum_{i=1}^N J^x \hat{\sigma}^x_i \hat{\sigma}^x_{i+1} +J^y  \hat{\sigma}^y_i \hat{\sigma}^y_{i+1} + J^z  \hat{\sigma}^z_i \hat{\sigma}^z_{i+1} + h  \hat{\sigma}^z_i\right).
\end{equation*}
If $J^x\neq J^y \neq J^z$, then it is $XYZ$ model, if $J^x= J^y \neq J^z=\Delta$, then it is XXZ model, and if $J^x= J^y=J^z$, then it is XXX model.

We concentrate on XXZ model,
\begin{equation}
	\mathcal{H} = \frac12 \left( \sum_{i=1}^N J \hat{\sigma}^x_i \hat{\sigma}^x_{i+1} +J  \hat{\sigma}^y_i \hat{\sigma}^y_{i+1} + \Delta \hat{\sigma}^z_i \hat{\sigma}^z_{i+1} + h  \hat{\sigma}^z_i\right),
\end{equation}
here $\Delta$ is the anisotropy term. The third term of the above equation commutes with the `z' component of the Pauli matrices and it preserves the eigenstates. 
Although for the first and second terms we will need to apply the unitary transformation and RWA to get the effective Hamiltonian. We can apply the RWA method in previous sections. Which will eventually results in `Localization' at $4h/\omega$ resonance points.
\end{document}