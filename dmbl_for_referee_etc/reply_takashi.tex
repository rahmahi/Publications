\documentclass[10pt]{article}

\usepackage{amsmath,amsthm,amssymb}
\usepackage{graphicx}% Include figure files
\usepackage{dcolumn}% Align table columns on decimal point
\usepackage{bm}% bold math
\usepackage{color}
\usepackage{epsfig}
\usepackage{multirow}
\usepackage{mathrsfs}
\usepackage{hyperref}
\usepackage{cleveref}
\usepackage{epstopdf}
\usepackage{subfigure}
\usepackage{autobreak}
\usepackage{todonotes}
\usepackage{physics}
\usepackage{bbm}
\usepackage[normalem]{ulem}


\usepackage[absolute,overlay]{textpos}

%Macros for mathematical notations

\newcommand{\V}[1]{\boldsymbol{#1}} %# vector
\newcommand{\M}[1]{\boldsymbol{#1}} %# matrix
\newcommand{\Set}[1]{\mathbb{#1}} %# set
\newcommand{\D}[1]{\Delta#1} %# \D{t} for time step size
\renewcommand{\d}[1]{\delta#1} %# \d{t} for small increment
\newcommand{\av}[1]{\left\langle #1\right\rangle } %take average

\newcommand{\sM}[1]{\M{\mathcal{#1}}} %matrix in mathcal font
\newcommand{\dprime}{\prime\prime} % double prime
%\global\long\def\i{\iota}
%\renewcommand{\i}{\iota} %i for imaginary unit
%\renewcommand{\i}{\mathsf i} %i for imaginary unit
\newcommand{\follows}{\quad\Rightarrow\quad} %=>
\newcommand{\eqd}{\overset{d}{=}} %=^d
\newcommand{\spe}[1]{\mathscr{#1}}  %important quantities in mathscr font
\newcommand{\eps}{\epsilon}

\newcommand{\response}[1]{{\color{black}#1}} % for authors' response
\newcommand{\comment}[1]{{\color{blue}#1}} % for referee's comment


\title{Reply to TM}

\begin{document}
\maketitle
We agree that the summation sign in equation (25), $\sum_{i<j}$, should be written in subscript notation. This will definitely provide more clarity to the spin-bond count in the manuscript.

We went over the KacNorm in equation 25. More investigation revealed a small discrepancy between the Hamiltonian in our referee response, the one utilized in our simulations and analytical computations, and the one given in the manuscript. In our simulations and computations, we employed the Hamiltonian 
\begin{equation}
	H = \sum_{i<j} \frac{2}{N-1} S^z_i S^z_j + 2h(t) \sum_i S^x_i, \label{eq:hams} 
\end{equation}
where $\vec{S}=\frac12\vec{\sigma}$. This was selected since rotation operations on spins become boilerplate algebra and qutip contains built-in algorithms to generate angular momentum matrix members. However, 
\begin{equation} 
	H = \frac{1}{2(N-1)} \sum_{i<j} \hat{\sigma}^z_i \hat{\sigma}^z_j + h(t) \sum_i \hat{\sigma}^x_i. \label{eq:hamh} 
\end{equation} 
is the result of expressing the Hamiltonian in terms of the $\sigma s$.

We had presented the Hamiltonian using the $\sigma s$ when we were drafting the paper, but we neglected to alter the KacNorm back to that in the $\sigma$ form in the first section of Section 2 eq. 25. Now, we have $\vec{S}$ in place of $\sigma s$ in equation (25), and it is similar to equation (\eqref{eq:hams}). The change maintains all other information while guaranteeing that the analysis and charts in the manuscript and reviewer response are left intact.

\noindent I've enclosed revised drafts of the manuscript and response. I'd like to send them off soon, so plz. let me know if you have any other questions.
\end{document}
