\documentclass[showpacs,floatfix,superscriptaddress, onecolumn, nofootinbib, 10pt]{revtex4-2}

\usepackage{amsmath,amsthm,amssymb}
\usepackage{graphicx}% Include figure files
\usepackage{dcolumn}% Align table columns on decimal point
\usepackage{bm}% bold math
\usepackage{color}
\usepackage{epsfig}
\usepackage{multirow}
\usepackage{mathrsfs}
\usepackage{hyperref}
\usepackage{cleveref}
\usepackage{epstopdf}
\usepackage{subfigure}
\usepackage{autobreak}
\usepackage{todonotes}
\usepackage{physics}
\usepackage{bbm}
\usepackage[normalem]{ulem}


\usepackage[absolute,overlay]{textpos}

%Macros for mathematical notations

\newcommand{\V}[1]{\boldsymbol{#1}} %# vector
\newcommand{\M}[1]{\boldsymbol{#1}} %# matrix
\newcommand{\Set}[1]{\mathbb{#1}} %# set
\newcommand{\D}[1]{\Delta#1} %# \D{t} for time step size
\renewcommand{\d}[1]{\delta#1} %# \d{t} for small increment
\newcommand{\av}[1]{\left\langle #1\right\rangle } %take average

\newcommand{\sM}[1]{\M{\mathcal{#1}}} %matrix in mathcal font
\newcommand{\dprime}{\prime\prime} % double prime
%\global\long\def\i{\iota}
%\renewcommand{\i}{\iota} %i for imaginary unit
%\renewcommand{\i}{\mathsf i} %i for imaginary unit
\newcommand{\follows}{\quad\Rightarrow\quad} %=>
\newcommand{\eqd}{\overset{d}{=}} %=^d
\newcommand{\spe}[1]{\mathscr{#1}}  %important quantities in mathscr font
\newcommand{\eps}{\epsilon}

\newcommand{\response}[1]{{\color{black}#1}} % for authors' response
\newcommand{\comment}[1]{{\color{blue}#1}} % for referee's comment


\title{reply to Mori}

\begin{document}

We concur with your suggestion to use subscript notation for the summation symbol in equation (25), namely $\sum_{i<j}$. This will undoubtedly give the spin-bond count in the manuscript greater clarity.

We reviewed the KacNorm in the eq.25. Upon inquiry, it was found that a disparity exists between the Hamiltonians presented in the manuscript and the referee's response. The Hamiltonian used in our numerical codes has been found to be,
\begin{equation}
    H = \frac{2}{N-1} \sum_{i<j} S^z_i S^z_j + 2h(t) \sum_i S^x_i.
    \label{eq:hams}
\end{equation}  
This can be expressed mathematically in Hilbert space by utilizing the Pauli spin matrix,
\begin{equation}
H = \frac{1}{2(N-1)} \sum_{i<j} \hat{\sigma}^z_i \hat{\sigma}^z_j + h(t) \sum_i \hat{\sigma}^x_i.
\label{eq:hamh}
\end{equation}
Nevertheless, in the text, in the initial part of Section 2 eq.25, we have formulated the Hamiltonian using Pauli matrices, and the KacNorm deviates from eq. \eqref{eq:hamh}. In order to resolve this inconsistency, we substituted the Pauli matrices in eq.(25) with the total spin operator $\vec{S}$ as stated in eq.\eqref{eq:hams}, which is closely comparable to the LMG Hamiltonian employed in the aforementioned equation which can be found at, https://doi.org/10.1103/PhysRevE.107.044130.

The modification ensures that the system dynamics and graphical plots in manuscripts remain unaltered, while preserving the vital information.


\noindent I've enclosed revised drafts of the manuscript and response. Please review them and provide any feedback if any.
\end{document}
